\section{BPE: Business Processes}

\subsection{Overview}
BPE is a part of Synrc business application stack that
unlocks Erlang for for enterprise core processing.
It provides infrastructure for workflow definitions, process orchestration,
rule-based production systems and distributed storage. This book is dedicated to cover
all parts needed for bootstrapping operational document processing model along with
workflow, models, forms, validations and other aspects of systems with similar requirements.

\subsection{Fullstack Business Applications}
This book is also about the foundations of banking implementation (TPS) up to external
connectors and services which is closed by security policies. So it also defines
transactional processing model, storage model, and distribution capabilities such as
Dynamo hashing to achieve data locality for transactions and documents which
belongs to bank customers.

\subsection{FORMS: Applications}
The FORMS application is dedicated to bring forms under the common
ontological model of data that are entering, storing and processing.
The forms model gives a root to the essence of information system circulation.

\subsection{BPE: Workflows}
The core of document processing is BPE which runs in native Erlang process semantic
as Pi-calculus execution environment. Different bank departments represented
as ACT workflow scenarios, such as deposit, credit or operational processes along with
transfer and withdraw or charge operations. All these operations represented
as document forms which flows along workflow. Activity service knows only
customer information and does not know card numbers or other sensitive transactional data.

\subsection{TPS: Processing}
The monetary processing core is TPS service driven by scripts in UPL language.
All constraints and rules are applied from UPL definitions to transactions.
Transactional service knows nothing about customer information.

\subsection{CR: Database}
The basic database schema DBS to work with BPE and TPS applications.
There are two implementations of TPS service: one is using {\bf riak\_core}
database ifrastructure, the other is based on chain replication CR database
using {\bf kvs} database infrastructure and rafter protocol as oracul.
The CR database is based on Robert van Renessee papers. CR has also
XA protocol capabilities.

\subsection{UPL: Language}
UPL allows to define transaction tarification rules in human readable form.
