% copyright (c) 2018 Groupoid Infinity

\documentclass{article}
\usepackage[english,ukrainian]{babel}
\usepackage{hyphenat}
\usepackage{listings}
\usepackage{amsmath}
\usepackage{amssymb}
\usepackage{amsthm}
\usepackage{mathtools}
\usepackage{url}
\usepackage{tikz-cd}
\usepackage[utf8]{inputenc}
\theoremstyle{definition}
\newtheorem{theorem}{Теорема}
\newtheorem{definition}{Визначення}
\newtheorem{exercise}{Вправа}
\newtheorem{example}{Приклад}
\newcommand*{\incmap}{\hookrightarrow}
\newcommand*{\thead}[1]{\multicolumn{1}{c}{\bfseries #1}}
\lstset{basicstyle=\small,inputencoding=utf8}

\addto\captionsukrainian{\renewcommand{\contentsname}{Зміст}}
\addto\captionsukrainian{\def\refname{Список використаних джерел}}
\addto\captionsukrainian{\renewcommand{\abstractname}{Анотація}}

\begin{document}
\title{Anders: гомотопічна бібліотека}
\author{Максим Сохацький $^{1,2}$}
\date{ \small $^1$ Національний технічний університет України \\
       ім. Ігоря Сікорського \\
       \small $^2$ Інститут математики «Групоїд Інфініті» \\
       26 листопада 2023 }
\maketitle

\begin{abstract}
Тут представлена базова гомотопічна бібліотека мови {\bf Anders} для курсу «Теорія типів»,
яка сумісна з позначеннями, що використовуються в підручнику HoTT.
Серед принципів, які покладені в основу бібліотеки, головними є:
лаконічность, академічність, педагогічність. Кожна сторінка має
на меті повністю висвітлити компоненти типу, використовуючи тільки
ті типи, що були викладені попередньо, кожне визначення повинно
містити як математичну нотацію так і код верифікатора та бути
вичерпним посібником користувача мови програмування {\bf Anders} та
її базової бібліотеки. Загалом передбачається, що бібліотека
повинна відповідати підручнику HoTT, та бути його практичним
досліднидницьким артефактом.
\\
\\
{\bf Ключові слова}: Теорія типів, формалізація математики
\end{abstract}

\section*{Теорія типів}
Теорія типів --- це універсальна мова програмування чистої
математики (для доведення теорем), яка може містити довільну
кількість консистентних аксіом, впорядкованих у вигляді псевдо-ізоморфізмів:
1) сигнатури типу або формації;
2) функції encode, способи конструювання елементів типу або конструкція;
3) функції decode, залежні елімінатори принципу індукції типу або елімінація;
4) рівняння бета правила або обчислювальності;
5) рівняння ета правила або унікальності.
Таке визначення було дано Мартіном-Льофом,
від чого теорія типів носить його ім'я MLTT.

Головна мотивація гомотопічної теорії — надати обчислювальну
семантику гомотопічним типам та CW-комплексам. Головна ідея
гомотопічної теорії [1] полягає в поєднанні просторів функцій,
просторів контекстів  і просторів шляхів  таким чином, що вони
утворюють фібраційну рівність яка збігається (доводиться в самій
теорії) з простором шляхів.

Завдяки відсутності ета-правила у рівності, не кожні два
доведення одного простору шляхів дорівнюють між собою, отже
простір шляхів утворює багатовимірну структуру інфініті-групоїда.

\newpage
\section*{Основи}

Перша частина базової бібліотеки --- модальні унівалентні MLTT основи, що розділені на три групи.
Перша група містить класичні типи MLTT системи описані Мартіном-Льофом,
які присутні у мовах $\mathbf{Per}$ та $\mathbf{Anders}$.
Друга група містить унівалентні ідентифікаційні системи мови $\mathbf{Anders}$.
Третя група містить модальності мови $\mathbf{Anders}$, які використовуються в диференціальній геометріїї та в теорії гомотопій.
Основи пропонують фундаментальний базис який використовується
для формалізації сучасної математики в таких системах доведення теорем як: Coq, Agda, Lean.\\
\\
\noindent
$\bullet$ Фібраційні \\
$\bullet$ Унівалентні \\
$\bullet$ Модальні

\section*{Математики}

Друга частина базової бібліотеки $\mathbf{Anders}$ містить формалізації математичних
теорій з різних галузей математики: аналіз, алгебра, геометрія,
теорія гомотопій, теорія категорій.

Слухачам курсу (10) пропонується застосувати теорію типів для
доведення початкового але нетривіального результу, який є
відкритою проблемою в теорії типів для однєї із математик,
що є курсами на кафедрі чистої математики (КМ-111):\\
\\
\noindent
$\bullet$ Функціональний аналіз \\
$\bullet$ Гомологічна алгебра \\
$\bullet$ Диференціальна геометрія \\
$\bullet$ Теорія гомотопій \\
$\bullet$ Теорія категорій \\

\newpage
\section*{Програми}

Третя частини базової бібліотеки, присутня у мовах $\mathbf{Per}$ та $\mathbf{Anders}$,
присвячена прикладам з промислового програмування в області автоматизації
підприємств та інформаційних технологій, а саме для специфікації програмних інтерфейсів.
\\
\\
\noindent $\bullet$ Формалізація двонаправленого тракту \\
$\bullet$ Формалізація графічного веб інтерфейсу \\
$\bullet$ Формалізація бази даних з єдиним простором ключів \\
$\bullet$ Формалізація реляційної бази даних \\
$\bullet$ Формалізація системи управління процесами \\

\section*{Філософії}

З сучасників формальною філософією в HoTT загалом займається Девіт Корфілд,
а формалізацією свідомості як окремий предмет вивчають Хенк Барендрехт та Горо Като.
Формальна теорія природніх мов теж формалізується за допомогою MLTT,
а основні теореми доводять в HoTT. В четвертій частині базової
бібліотеки $\mathbf{Anders}$ наводяться приклади програм, які
маніфестують висловлювання і теореми з формальної філософії про пустотність
всіх феноменів та синтаксис, морфологію і семантику природньої української мови.
\\
\\
\noindent $\bullet$ Формалізація Мадг'яміки \\
$\bullet$ Формалізація української мови в кванторах

\newpage
\section*{Структура верифікатора}

На відміну від одноаксіоматичного верифікатора $\mathbf{Henk}$, який містить тільки
один індексований всесвіт $U_i$, рівність за визначенням для примітивів єдиного $\Pi$-типу,
та функція верифікації $\mathbf{type}$, верифікатори $\mathbf{Per}$ і $\mathbf{Anders}$ містять додатково
$\Sigma$-тип для контекстів та телескопів, більш деталізовану функцію типізації $\tau$,
та багато інших досніпових модулів, крім $\Pi$-типу, але які теж підпорядковуються системі типів Мартіна-Льофа.

\subsection*{Космос $\mathbb{N}$-індексованих всесвітів $\omega$}

В теорії типів всі сигнатури всіх типів живуть в ієрархіях
всесвітів індексованих натуральними числами. Множина таких
ієрархій називається космосом. В імплементаціях $\mathbb{N}$
завжди реалізовано як Big Integer. Верифікатор $\mathbf{Anders}$ має
наступний космос $\omega = \{ \mathbf{U}_i, \mathbf{V}_i \}$.

\subsection*{Рівність $=_{def}$ з точністю до $\alpha$-$\beta$ конверсій}

Рівність за визначенням двох термів означає, що за допомогою серії альфа та бета
перетворень можна довести що терми дорівнюють посимвольно. Саме ця функція повинна бути
імплементована для всіх типів у верифікаторі. Програми, які доводять рівність двох термів
в теорії самого верифікатора за допомогою $=$-тип чи інших ідентифікаційних систем, як $\equiv$-типи чи інші,
називаються пропозиціональними рівностями.

\subsection*{Функція верифікації $\tau$}

Головна функція верифікації  розпадається на систему взаємозалежних функцій
$\tau = \{ \mathbf{infer},\mathbf{app}, \mathbf{check}, \mathbf{act}, \mathbf{conv}, \mathbf{eval} \}$,
які повинні бути імплементовані для кожного типу, вбудованого в верифікатор.

\subsection*{Контексти та телескопи $\mathrm{\Sigma}$}

В теорії типів контексти, як алгебраїчні послідовності які містять сигнатури, які теж у свою
чергу складаються з послідовністей пар, що складаються з імені змінної та її типу, визначаються
$\Sigma$-типами.

\subsection*{Досніпові модулі $\mathrm{\int}$ вбудованих типів}

Кожен досніповий модуль повинен бути представлений у вигляді п'яти
синтаксичних примітивів: 1) формації; 2) конструкції; 3) елімінації;
4) обчислювальності; 5) унікальності. Ці примітиви повинні бути узгоджені
в сенсі Мартіна-Льофа та представлені у цій статті,
як документація на бібліотеку верифікатора, як у тому числі дає формальне визначення
примітивам в конкретній теорії
$\mathrm{\int} = \{ \Pi, \Sigma, =, \mathbf{W}, \mathbf{0}, \mathbf{1}, \mathbf{2}, \mathbf{Path}, \mathbf{Glue} \}$.

\newpage

\section{Простори функцій}
$\Pi$-тип — це простір, що містить залежні функції, кодомен яких залежить від значення
з домену. Так як всі розшарування домену присутні повністю в кожній функції з простору,
$\Pi$-тип також називається залежним добутком, так як фунція визначена на всьому просторі домена.

Простори залежних функції використовуються в теорії типів для моделювання різних
математичних конструкцій, об'єктів, типів, просторів, а також їхніх відображень:
залежних функцій, неперервниї відображень, етальних відображень, розшарувань, квантора
узанальнення $\forall$, імплікації, тощо.

 \subsection{Формація}

$\textbf{Визначення\ 1.1}$ ($\Pi$-формація, залежний добуток).
$\Pi$-типи репрезентують спосіб створення просторів залежних функцій  $f: \Pi(x:A), B(x)$ в певному всесвіті $U_i$,
з доменом в $A$ і кодоменом в сім'ї функцій $B : A \rightarrow U_i$ над $A$.
$$
   \Pi : U =_{def} \prod_{x:A}B(x).
$$
\begin{lstlisting}[mathescape=true]
def Pi (A : U) (B : A $\rightarrow$ U) : U
 := $\Pi$ (x : A), B(x)
\end{lstlisting}

\subsection{Конструкція}

$\textbf{Визначення\ 1.2}$ ($\lambda$-функція).
Лямбда конструктор визначає нову лямбда функцію в просторі залежних функцій,
вона ще називається лямбда абстракцією і позначається як $\lambda x. b(x)$ або $x \mapsto b(x)$.
$$
    \lambda (x: A) \rightarrow b(x) : \Pi(A,B) =_{def}
$$
$$
    \prod_{A:U}\prod_{B:A \rightarrow U}\prod_{a: A}\prod_{b:B(a)}\lambda x.b.
$$
\begin{lstlisting}[mathescape=true]
def lambda (A: U) (B: A $\rightarrow$ U) (b: Pi A B)
  : Pi A B := $\lambda$ (x : A), b(x)

def lam (A B: U) (f: A $\rightarrow$ B)
  : A $\rightarrow$ B := $\lambda$ (x : A), f(x)
\end{lstlisting}

Коли кодомен не залежить від значеення з домену функції $f: A \rightarrow B$
розглядаються в контексті System F$_\omega$, залежний випадок розглядається
в  Systen P$_\omega$ або Calculus of Construction (CoC).

\newpage
\subsection{Елімінація}

$\textbf{Визначення\ 1.3}$ (Принцип індукції). Якшо предикат виконується для
лямбда функції тоді існує функція з простору функцій в простіп предикатів.

\begin{lstlisting}[mathescape=true]
def $\Pi$-ind (A : U) (B : A $\rightarrow$ U) (C : Pi A B $\rightarrow$ U)
    (g: $\Pi$ (x: Pi A B), C x)
  : $\Pi$ (p: Pi A B), C p := $\lambda$ (p: Pi A B), g(p)
\end{lstlisting}

\noindent $\textbf{Визначення\ 1.3.1}$ ($\lambda$-аплікація).
Застосування функції до аргументів редукує терм
використовуючи рекурсивну підстановку аргументів в тіло функції.

$$
    f\ a : B(a) =_{def} \prod_{A:U}\prod_{B: A \rightarrow U}\prod_{a:A}\prod_{f: \prod_{x:A}B(a)}f(a).
$$

\begin{lstlisting}[mathescape=true]
def apply (A: U) (B: A $\rightarrow$ U) (f: Pi A B) (a: A) : B a := f(a)
def app (A B: U) (f: A $\rightarrow$ B) (x: A): B := f(x)
\end{lstlisting}

\noindent $\textbf{Визначення\ 1.3.2}$ (Композиція функцій).
\begin{lstlisting}[mathescape=true]
def $\circ^T$ (x y z: U) : U
 := (y $\rightarrow$ z) $\rightarrow$ (x $\rightarrow$ y) $\rightarrow$ (x $\rightarrow$ z)

def $\circ$ (x y z : U) : $\circ^T$ x y z
 := $\lambda$ (g: x $\rightarrow$ z) (f: x $\rightarrow$ y) (a: x), g (f a)
\end{lstlisting}

\subsection{Обчислювальність}

$\textbf{Теорема\ 1.4}$ (Обчислювальність $\Pi_\beta$).
$\beta$-правило показує, що композиція $\mathrm{lam} \circ \mathrm{app}$ може бути скорочена (fused).
$$
    f(a) =_{B(a)} (\lambda (x:A) \rightarrow f(a))(a).
$$
\begin{lstlisting}[mathescape=true]
def $\Pi$-$\beta$ (A : U) (B : A $\rightarrow$ U) (a : A) (f : Pi A B)
  : Path (B a) (apply A B (lambda A B f) a) (f a)
 := idp (B a) (f a)
\end{lstlisting}

\subsection{Унікальність}

$\textbf{Теорема\ 1.5}$ (Унікальність $\Pi_\eta$).
$\eta$-правило показує, що композиація $\mathrm{app} \circ \mathrm{lam}$ можу бути скоронеча (fused).
$$
    f =_{(x:A)\rightarrow B(a)} (\lambda (y:A) \rightarrow f(y)).
$$
\begin{lstlisting}[mathescape=true]
def $\Pi$-$\eta$ (A : U) (B : A $\rightarrow$ U) (a : A) (f : Pi A B)
  : Path (Pi A B) f ($\lambda$ (x : A), f x)
 := idp (Pi A B) f
\end{lstlisting}

\newpage
\section{Простори контекстів}

$\Sigma$-тип --- це простір, що містить залежні пари, де тип другого
елемента залежить від значення першого елемента. Оскільки в кожній
визначеній парі присутня лише одна точка домену волокна, — тип також
є залежною сумою, де основа волокна є непересічним об'єднанням.

Простори залежних пар використовуються в теорії типів для моделювання
декартових добутків, непересічних сум, розшарувань, векторних просторів,
телескопів, лінз, контекстів, об'єктів, алгебр, квантору існування $\exists$, тощо.

\subsection{Формація}

$\textbf{Визначення\ 2.1}$ ($\Sigma$-формація, залежна сума). Тип залежної суми
індексований типом  $A$ в сенсу кодобутку або диз'юнктивної суми, де тільки одне
волокно кодомену $B(x)$ присутнє в парі.
$$
     \Sigma : U =_{def} \sum_{x:A} B(x).
$$
\begin{lstlisting}[mathescape=true]
def Sigma (A: U) (B: A $\rightarrow$ U) : U
 := $\Sigma$ (x: A), B(x)
\end{lstlisting}

\subsection{Конструкція}

$\textbf{Визначення\ 2.2}$ (Залежна пара). Конструктор залежної пари —
це спосіб визначення індексованої пари над типом $A$ елементу кодобутку
або диз'юнктивного об'єднання.
$$
      \mathbf{pair} : \Sigma(A,B) =_{def}
$$
$$
      \prod_{A:U}\prod_{B:A \rightarrow U}\prod_{a:A}\prod_{b:B(a)} (a,b).
$$
\begin{lstlisting}[mathescape=true]
def pair (A: U) (B: A $\rightarrow$ U) (a: A) (b: B a)
  : Sigma A B := (a, b)
\end{lstlisting}

\newpage

\subsection{Елімінація}

$\textbf{Визначення\ 2.3}$ (Проекції). Залежні проекції
$pr_{1}: \Sigma(A,B) \rightarrow A$ і
$pr_{2}: \Pi_{x: \Sigma(A,B)} B(pr_{1}(x))$ є деконструкторами пари.
$$
    \mathbf{pr}_1 : \prod_{A:U} \prod_{B:A \rightarrow U} \prod_{x: \Sigma(A,B)} A
$$
$$
    =_{def} .1 =_{def} (a,b) \mapsto a.
$$
$$
    \mathbf{pr}_2 : \prod_{A:U} \prod_{B:A \rightarrow U} \prod_{x: \Sigma(A,B)} B(x.1)
$$
$$
    =_{def} .2 =_{def} (a,b) \mapsto b.
$$
\begin{lstlisting}[mathescape=true]
def pr$_1$ (A: U) (B: A $\rightarrow$ U) (x: Sigma A B) : A := x.1
def pr$_2$ (A: U) (B: A $\rightarrow$ U) (x: Sigma A B) : B (pr$_1$ A B x) := x.2
\end{lstlisting}

Якшо ви хочете доступитися до глибокого (>1) поля в сігма-типі — ви повинні використати
серію елімінаторів $.2$, яка закінчується елімінатором $.1$.
\\
\\
\noindent $\textbf{Визначення\ 2.3.1}$ (Принцип індукції $\Sigma$). Каже, що
предикат, який виконується для двох проекцій, він виконується також і для
всього простору пар.

\begin{lstlisting}[mathescape=true]
def $\Sigma$-ind (A : U) (B : A $\rightarrow$ U)
    (C : $\Pi$ (s: $\Sigma$ (x: A), B x), U)
    (g: $\Pi$ (x: A) (y: B x), C (x,y))
    (p: $\Sigma$ (x: A), B x) : C p := g p.1 p.2
\end{lstlisting}

\subsection{Обчислювальність}

$\textbf{Визначення\ 2.4}$ ($\Sigma$-обчислювальність).
\begin{lstlisting}[mathescape=true]
def $\Sigma$-$\beta_1$ (A : U) (B : A $\rightarrow$ U) (a : A) (b : B a)
  : Path A a (pr$_1$ A B (a, b)) := idp A a

def $\Sigma$-$\beta_2$ (A : U) (B : A $\rightarrow$ U) (a : A) (b : B a)
  : Path (B a) b (pr$_2$ A B (a, b)) := idp (B a) b
\end{lstlisting}

\subsection{Унікальність}

$\textbf{Визначення\ 2.5}$ ($\Sigma$-унікальність).
\begin{lstlisting}[mathescape=true]
def $\Sigma$-$\eta$ (A : U) (B : A $\rightarrow$ U) (p : Sigma A B)
  : Path (Sigma A B) p (pr$_1$ A B p, pr$_2$ A B p)
 := idp (Sigma A B) p
\end{lstlisting}

\newpage
\section{Ідентифікаційні простори}

$=$-тип --- це індуктивна родина функій індексована елементами $x,y : A$,
які містять доведення того факту, що ці елементи рівні між собою $x=y$.

\subsection{Формація}
$\mathbf{Definition\ 3.1}$ ($=$-формація, родина залежних функцій). Індуктивна родина
$Id_V: A \rightarrow A \rightarrow V$ з доменом і кодоменом у всесвіті $V$
представляє елементи, що містять доведення факту, що індексовані $x,y:A$
елменти рівні між собою.
$$
  =\hspace{0.4em} : U =_{def} \prod_{A:V}\prod_{x,y:A} \mathbf{Id}_V(A,x,y).
$$
\begin{lstlisting}[mathescape=true]
def IdV (A: V) (x y: A)
  : V := Id A x y
\end{lstlisting}


\end{document}

