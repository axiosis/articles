% copyright (c) 2018 Groupoid Infinity

\documentclass{article}
\usepackage[english,ukrainian]{babel}
\usepackage{listings}
\usepackage[utf8]{inputenc}

\addto\captionsukrainian{\renewcommand{\contentsname}{Зміст}}
\addto\captionsukrainian{\renewcommand{\abstractname}{Аннотація}}

\begin{document}

\author { М.Е. Сохацький$^1$ }
\title { Per: інтерналізація MLTT-73 }
\date { \small Групоїд Інфініті. \\ \small $^*$ Кореспондент: namdak@tonpa.guru }
\maketitle

\begin{abstract}

Ця стаття демонструє формальне вбудовування моделі теорії типів Мартіна-Льофа
в виконуючу кубічну типову систему з мінімальним набором правил виводу, яка називається мовою {\bf Per}.
Був пройдений довгий шлях від чистих типових систем AUTOMATH де Брейна до
гомотопічних типових верифікаторів. Ця стаття стосується тільки формального ядра
теорії типів Мартіна-Льофа: $\Pi$ и $\Sigma$ типів (які відповідають
квантору загальності $\forall$ та квантору існування $\exists$ у класичній логіці)
та типу-рівності.

Кожна мовна імплементація повинна бути протестована. Один з можливих сценаріїв
тестування типових верифікаторів це пряме вбудовування в модель теорії типів
виконуючого верифікатора. Так як всі типи в теорії формулюються за допомогою п'яти
прарвил: формації, конструкціїї, елімінації, обчислювальності, унікальності), ми зконструювали
номінальні типи-синоніми для виконуючого верифікатора та довели, що це є реалізацією MLTT.
Це може розглядатися як універсальний тест для імплементації типового верифікатора,
позаяк компенсаця інтро правила та правила елімінатора пов'язані в правилі
обчислення та рівності (бета та ета редукціях). Таким чином, доводжучи реалізацію MLTT,
ми доводимо властивості самого виконуючого верифікатора.

Більш формально, кубічне MLTT вбудовування конструктивно виражає
J елімінатор типу-рівності та його рівняння — правило обчислення,
що було неможливо до кубічної інтерпретації. Також цей випуск
відкриває серію статей присвячених формалізації основ математики в кубічній теорії типів,
MLTT моделюванню та кубічнії верифікації. Так як не всі можуть бути знайомі з теорією типів,
це випуск також містить їх інтерпретації з точки зору різних розділів математики.

\end{abstract}
\end{document}